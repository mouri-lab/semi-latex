\section{はじめに}
はじめには,2ページを目安に書きましょう.
卒論は量が多いため,チェックする人は変更点を見つけるのが大変です.
指摘して頂いた点は\verb|\Ca{ }|\Ca{で囲むことで文字が赤色になります}.
提出時(黒に戻すとき)は,main.texの\verb|\setcounter{ChangedColor}{0}|を0から1にしてください.

参考文献はbibtexを使いましょう.普段からゼミで使用している人は,referencesファイルを自分のものに
置き換えてください.
bibtexの使い方は,references.bibを作り,\verb|\cite{jmoni}|の様に本文で参照\cite{jmoni}し,
jbibtexコマンドでさくっとできます.
論文データベースには,必ずbibtex形式というのが用意されているはず.
その内容をコピーすれば基本は大丈夫.
参考文献のスタイルは,情報処理学会の出現順のものを使用しています.
