\documentclass[dvipdfmx,titlepage,10pt]{jarticle}

\usepackage[dvipdfmx]{graphicx}
\usepackage{latexsym}
\usepackage{url}
\usepackage{setspace}
\usepackage{svg}
\usepackage{here}
\usepackage{lipsum}
\usepackage{enumerate}
\usepackage{multirow}
\usepackage{booktabs}
\usepackage{threeparttable}
\usepackage{seqsplit}
% \usepackage{master-cover}
\usepackage{master_thesis}
\usepackage{master}
\usepackage[nobreak]{cite}

\usepackage{listings, jlisting} 		% for source code

\lstset{
	%プログラム言語(複数の言語に対応,C,C++も可)
 	% language = C++,
 	%枠外に行った時の自動改行
 	breaklines = true,
 	%自動改行後のインデント量(デフォルトでは20[pt])
 	breakindent = 12pt,
 	%標準の書体
 	basicstyle = \ttfamily\scriptsize,
 	%コメントの書体
 	commentstyle = {\itshape \color[cmyk]{1,0.4,1,0}},
 	%関数名等の色の設定
 	% classoffset = 0,
 	%キーワード(int, ifなど)の書体
 	% keywordstyle = {\bfseries \color[cmyk]{0,1,0,0}},
 	%表示する文字の書体
 	stringstyle = {\ttfamily \color[rgb]{0,0,1}},
 	%枠 "t"は上に線を記載, "T"は上に二重線を記載
	%他オプション:leftline,topline,bottomline,lines,single,shadowbox
 	frame = ltbr,
	%行番号の位置
	% numbers = left,
	%行番号の間隔
 	stepnumber = 1,
	%タブの大きさ
 	tabsize = 4,
}

\usepackage[a4paper,textheight=21.7cm,top=3cm,headheight=15pt]{geometry}
% \usepackage[a4paper,textheight=21.7cm,top=3cm,left=2cm,right=2cm,headheight=15pt]{geometry}


\setcounter{secnumdepth}{6}
\makeatletter
\renewcommand{\paragraph}{\@startsection{paragraph}{4}{\z@}%
   {1.5\Cvs \@plus.5\Cdp \@minus.2\Cdp}%
   {.5\Cvs \@plus.3\Cdp}%
   {\reset@font\normalsize\bfseries}}
\makeatother

\makeatletter
\newcommand{\subsubsubsection}{\@startsection{paragraph}{4}{\z@}%
  {1.0\Cvs \@plus.5\Cdp \@minus.2\Cdp}%
  {.1\Cvs \@plus.3\Cdp}%
  {\reset@font\sffamily\normalsize}
}
\makeatother
\setcounter{secnumdepth}{4}

\newcommand\figref[1]{図~\ref{#1}}
\newcommand\tabref[1]{表~\ref{#1}}



\begin{document}
\setlength{\baselineskip}{7mm} % 行間
\setlength{\oddsidemargin}{6mm}
\pagestyle{empty}

\年度{2025}
\題目{論文タイトル}
\指導教官{毛利 公一 教授}
\学籍番号{123456000-0}
\氏名{立命 太郎}
\コース{計算機科学コース} %計算機科学コース or 人間情報科学コース

\maketitle             % 表紙 (年度,タイトル,氏名,学籍番号を記述)
% 内容梗概

\begin{center}
  \large\bf 論文タイトル
\end{center}
\begin{flushright}
名前 書く
\end{flushright}

\vspace*{-12mm}
\section*{\normalsize 内容梗概}
本論文をまとめましょう.

          % 内容梗概(タイトル,氏名を記述)

% 目次(行間を修正して1ページに納めたいなどはこちら)
% {\setlength{\baselineskip}{17.2pt} \tableofcontents}
\tableofcontents            % 目次
\clearpage
\listoffigures              % 図目次
\listoftables               % 表目次
\clearpage

\setcounter{page}{1}
\pagestyle{myheadings}

\include{introduction}      % はじめに

\section*{謝辞}
\AddTableOfContents{謝辞}

最後に,日頃から励まし,応援して頂いた家族に心より感謝申し上げます.
   % 謝辞

\addcontentsline{toc}{section}{参考文献} %参考文献を目次に入れるやつ

\raggedright
{
	\bibliography{references}
	\bibliographystyle{ipsjunsrt}
}

\end{document}
