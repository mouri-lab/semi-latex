\documentclass{css}
%\documentclass[english]{css}

\usepackage[dvips]{graphicx}
\usepackage{latexsym}

\def\|{\verb|}

\newcommand{\cssyear}[0]{2024}
\newcommand{\cssname}[0]{CSS 2024}
\newcommand{\cssversion}[0]{2024/06/01}
\newcommand{\cssemail}[0]{css2024-office@iwsec.org}

\begin{document}

%% 本文が和文の場合,タイトル・著者名・著者所属・概要は,和文・英文共に必須.
%% If you prepare this manuscript in English, there is no need to put Japanese metadata (title, author names, affiliations, abstract, and keywords) in it.

\title{\cssname 電子投稿案内 (\cssversion 版)}
\etitle{Instruction for \cssname~Electronic Submission}

\affiliate{XX}{XX大学コンピュータ研究所\\
Institute of Computer, XX University}
\affiliate{YY}{株式会社YYセキュリティ研究所\\
Security Laboratories, YY Corporation}
\paffiliate{ZZ}{国立研究開発法人ZZ研究所\\
National Institute of ZZ}

%% メールアドレスは省略可能だが,代表者のメールアドレスは必須.
%% 姓名の間は半角スペースを入れること.

\author{情報 太郎}{Taro Joho}{XX}[taro.joho@xx.ac.jp]
\author{安全 花子}{Hanako Anzen}{XX, YY, ZZ}

%% the following is author command for english option.
%% at least one e-mail address is required.

%\author{Taro Joho}{XX}[taro.joho@xx.ac.jp]
%\author{Hanako Anzen}{XX, YY, ZZ}

\begin{abstract}
本稿では \cssname における電子投稿要領について解説します.
\cssname では オンライン投稿でのみ原稿を受け付けています.
今後,予告なく情報が更新されることがありますので,投稿前に最新の情報をホームページにてご確認下さい.
\end{abstract}

%% キーワード (1--5単語) の記載は任意.

\begin{jkeyword}
\cssname,\LaTeX,スタイルファイル
\end{jkeyword}

\begin{eabstract}
This document describes an instruction to submit a camera-ready version of your manuscript for Computer Security Symposium \cssyear \ (\cssname).
All manuscripts must be submitted through an electronic submission form.
This instruction is subject to change without notice. 
Please confirm that you are referring to the latest information in our site when you submit.
\end{eabstract}

%% the following keyword part is optional and can be omitted.

\begin{ekeyword}
\cssname, \LaTeX, style files
\end{ekeyword}

%% if you use english opsion, you should put your English abstract in the abstract environment.
%% eabstract is not displayed in english mode.

\maketitle

%1
\section{はじめに}

\cssname では,電子投稿によりご提出いただいた原稿にラベルやページ番号追加などの編集作業を行います.
閲覧・検索・印刷などにおいて活用できるように,講演原稿には一定のフォーマットと制限が設けられています.
原稿作成ならびに原稿提出にあたりご一読をお願いいたします.

なお,CSS2016 より,\LaTeX のスタイルファイルを情報処理学会論文執筆キット\footnote{http://www.ipsj.or.jp/journal/submit/style.html} に基づく新しいもの (\|css.cls|) に刷新しています.
本稿では,そのスタイルファイルを用いた論文の作成方法に関して示します.

本スタイルファイルを利用した論文の作成方法などを情報処理学会に問い合わせることは\textbf{絶対に}ご遠慮ください.

%2
\section{基本的な使い方}

\cssname スタイルファイル (\|css.cls|) は,内部的には情報処理学会の論文執筆キットに含まれる \|ipsj.cls| を \|[submit,techrep]| オプション付きで呼び出す wrapper になっています.
そのため,\|css.cls| 利用時には,同梱の \|ipsj.cls| と \|ipsjtech.sty| も併せて同じディレクトリに配置した上で,\|\documentclass{css}| と宣言してください.
その後の使い方は\|ipsj.cls| とほぼ同じです.

なお,\|css.cls| には \|ipsj.sty| に由来する英文投稿用のオプションがあり,

\noindent
\|\documentclass[english]{css} |\\
のようにお使いください.
それ以外のオプションは警告の上で無視されます.

\section{論文の構成}

\cssname スタイルファイルを用いた論文の構成は \|ipsj.cls| を用いた論文の構成に準じ,\figref{fig:structure} のようになります.
\cssname からの重要な変更点として,\textbf{本文が和文の場合は,表題・著者名・著者所属・概要は,それぞれ和文・英文の両方を記載してください}.
本文が英文の場合は,それらの和文は不要ですので,英文のもののみの記載としてください.
キーワードの記載は任意です.

\begin{figure}[tb]
\noindent
\|\documentclass{css}| \\
\%\|\documentclass[english]{css}| \\[4pt]
\quad \|<必要ならばユーザのマクロをここに記述>| \\[4pt]
\|\begin{document}|\\[4pt]
%
\|\title{表題(和文)}|\\[4pt]
\|\etitle{表題(英文)}|\\[4pt]
%
\|\affiliate{所属ラベル}{<和文所属>\\<英文所属>}|\\
\quad \|<|必要ならば \|\paffiliate| により現在の所属を宣言\|>|\\
\|\paffiliate{現所属ラベル}{<和現所属>\\<英現所属>}|\\[4pt]
%
\|\author{情報 太郎}{Taro Joho}|\\
\|          {<所属ラベル>}[E-mail]|\\
\|\author{安全 花子}{Hanako Anzen}|\\
\|          {<所属ラベル2,現所属ラベル3>}|\\[4pt]
%
\|********************|\\
\%\textbf{englishオプションを利用する場合は以下の通り}\\
\|\author{Taro Joho}|\\
\|          {<所属ラベル>}[E-mail]|\\
\|\author{Hanako Anzen}|\\
\|          {<所属ラベル2,現所属ラベル3>}|\\
\|********************|\\[4pt]
%
\|\begin{abstract}|\\
\|<概要(和文)>|\\
\|\end{abstract}|\\[4pt]
%
\|\begin{jkeyword}|\\
\|<キーワード>|\\
\|\end{jkeyword}|\\[4pt]
%
\|\begin{eabstract}|\\
\|<概要(英文)>|\\
\|\end{eabstract}|\\[4pt]
%
\|\begin{ekeyword}|\\
\|<Keywords>|\\
\|\end{ekeyword}|\\[4pt]
%
\|\maketitle|\\[4pt]
%
\|\section{|第1節の表題\|}|\\
\dots\dots\dots\dots\dots\\
\quad \|<本文>|\\
\dots\dots\dots\dots\dots\\
\|\begin{acknowledgment}|\\
\|<謝辞> (謝辞がある場合)|\\
\|\end{acknowledgment}|\\[4pt]
%
\|\begin{thebibliography}{99}%9 or 99|\\
\|\bibitem{1}|\\
\|\bibitem{2}|\\
\|\end{thebibliography}|\\[4pt]
%
\%\textbf{以下は付録がある場合のみ}\\
\|\appendix|\\
\|\section{|付録1節の表題\|}|\\[4pt]
\|\end{document}|
\caption{css.cls を用いた論文の構成}
\ecaption{Structure of a paper using css.cls.}
\label{fig:structure}
\end{figure}

以下,構成要素のそれぞれについて,論文執筆キットの研究報告サンプルファイル (\|tech-jsample.tex|) の記載から抜粋して説明します.
特に,englishオプションを利用するかどうかで\|\author|の記述が変わることに留意してください.


\subsection{表題・著者名等}

表題,著者名とその所属,および概要を前述のコマンドや環境により定義した後,\|\maketitle| によって出力します.

\subsubsection{表題} 

表題は,\|\title| および \|\etitle| で定義した表題はセンタリングされます.
文字数の多いものについては,適宜 \|\\| を挿入して改行します.

\subsubsection{著者名・所属} 

各著者の所属を第一著者から順に \|\affiliate| を用いてラベル(第1引数)を付けながら定義すると,脚注に番号を付けて所属が出力されます.
なお,複数の著者が同じ所属である場合には,一度定義するだけでよいです.

現在の所属は \|\paffiliate| を用い,同様にラベル,所属先を記述します.
所属先には自動で「現在」,\|\\|の改行で「Presently with」が挿入されます.
著者名は \|\author| で定義します.各著者名の直後に,英文著者名,所属ラベルとメールアドレスを記入します.
著者が複数の場合は \|\author| を繰り返すことで,2人,3人,\dots と増えていきます.
現在の所属や,複数の所属先を追加する場合には,所属ラベルをカンマで区切り,追加すればよいです.

また,メールアドレス部分は省略が可能ですが,必ず代表者のアドレスは必要となります.
なお,和文著者名,英文著者名は,姓と名を半角(ASCII)の空白で区切ります.

\subsubsection{概要} 

和文の概要は \|abstract| 環境の中に,英文の概要は \|eabstract| 環境の中に,それぞれ記述します.


\subsubsection{キーワード} 

和文の概要は \|jkeyword| 環境の中に,英文の概要は \|ekeyword| 環境の中に,それぞれ1〜5語記述します.

\subsection{本文}

\subsubsection{見出し}

節や小節の見出しには \|\section|, \|\subsection|, \|\subsubsection|, \|\paragraph| といったコマンドを使用します.

\<「定義」,「定理」などについては,\|\newtheorem|で適宜環境を宣言し,その環境を用いて記述します.

\subsubsection{行送り}

2段組を採用しており,左右の段で行の基準線の位置が一致することを原則としています.
また,節見出しなど,行の間隔を他よりたくさんとった方が読みやすい場所では,この原則を守るようにスタイルファイルが自動的にスペースを挿入します.
したがって本文中では \|\vspace| や \|\vskip| を用いたスペースの調整を行なわないようにしてください.

\subsubsection{フォントサイズ}

フォントサイズは,スタイルファイルによって自動的に設定されるため,基本
的には著者が自分でフォントサイズを変更する必要はありません.

\subsubsection{句読点}

句点には全角の「.」,読点には全角の「,」を用います.ただし英文中や数式中
で「.」や「,」を使う場合には,半角文字を使います.「。」や「、」は使いません.

\subsubsection{全角文字と半角文字}

全角文字と半角文字の両方にある文字は次のように使い分けます.

\begin{enumerate}
\item 括弧は全角の「(」と「)」を用います.但し,英文の概要,図表見出し,書誌データでは半角の「(」と「)」を用います.

\item 英数字,空白,記号類は半角文字を用います.ただし,句読点に関しては,前項で述べたような例外があります.

\item カタカナは全角文字を用います.

\item 引用符では開きと閉じを区別します.
開きには \|``| を用い,閉じには\|''| を用います.
\end{enumerate}

\subsubsection{箇条書}

箇条書に関する形式を特に定めていません.場合に応じて標準的な \|enumerate|, \|itemize|, \|description| の環境を用いてよいです.

\begin{figure}[tb]
\setbox0\vbox{
\hbox{\|\begin{figure}[tb]|}
\hbox{\quad \|<|図本体の指定\|>|}
\hbox{\|\caption{<|和文見出し\|>}|}
\hbox{\|\ecaption{<|英文見出し\|>}|}
\hbox{\|\label{| $\ldots$ \|}|}
\hbox{\|\end{figure}|}
}
\centerline{\fbox{\box0}}
\caption{1段幅の図}
\ecaption{Single column figure with caption\\
explicitly broken by $\backslash\backslash$.}
\label{fig:single}
\end{figure}

\subsection{図}

1段の幅におさまる図は,\figref{fig:single} の形式で指定します.
位置の指定に \|h| は使いません.
また,図の下に和文と英文の双方の見出しを,\|\caption| と \|\ecaption| で指定します.
文字数が多い見出しは自動的に改行して最大幅の行を基準にセンタリングしますが,見出しが2行になる場合には適宜 \|\\| を挿入して改行したほうが良い結果となることがしばしばあります(\figref{fig:single} の英文見出しを参照).
図の参照は \|\figref{<|ラベル\|>}| を用いて行ないます.

\begin{figure}[tb]
\begin{minipage}[t]{0.5\columnwidth}
\footnotesize
\setbox0\vbox{
\hbox{\|\begin{minipage}[t]%|}
\hbox{\|  {0.5\columnwidth}|}
\hbox{\|\CaptionType{table}|}
\hbox{\|\caption{| \ldots \|}|}
\hbox{\|\ecaption{| \ldots \|}|}
\hbox{\|\label{| \ldots \|}|}
\hbox{\|\makebox[\textwidth][c]{%|}
\hbox{\|\begin{tabular}[t]{lcr}|}
\hbox{\|\hline\hline|}
\hbox{\|left&center&right\\\hline|}
\hbox{\|L1&C1&R1\\|}
\hbox{\|L2&C2&R2\\\hline|}
\hbox{\|\end{tabular}}|}
\hbox{\|\end{minipage}|}}
\hbox{}
\centerline{\fbox{\box0}}
\caption{\protect\tabref*{tab:right} の中身}
\ecaption{Contents of Table \protect\ref{tab:right}.}
\label{fig:left}
\end{minipage}%
\begin{minipage}[t]{0.5\columnwidth}
\CaptionType{table}
\caption{\protect\figref*{fig:left} で作成した表}
\ecaption{A table built by\\ Fig.\,\protect\ref{fig:left}.}
\label{tab:right}
\vskip1mm
\makebox[\textwidth][c]{\begin{tabular}[t]{lcr}\hline\hline
left&center&right\\\hline
L1&C1&R1\\
L2&C2&R2\\\hline
\end{tabular}}
\end{minipage}
\end{figure}

\begin{figure*}[tb]
\setbox0\vbox{\large
\hbox{\|\begin{figure*}[t]|}
\hbox{\quad \|<|図本体の指定\|>|}
\hbox{\|\caption{<|和文見出し\|>}|}
\hbox{\|\ecaption{<|英文見出し\|>}|}
\hbox{\|\label{| $\ldots$ \|}|}
\hbox{\|\end{figure*}|}}
\centerline{\fbox{\hbox to.9\textwidth{\hss\box0\hss}}}
\caption{2段幅の図}
\ecaption{Double column figure.}
\label{fig:double}
\end{figure*}

また紙面スペースの節約のために,1つの \|figure|(または \|table|)環境の中に複数の図表を並べて表示したい場合には,\figref{fig:left} と \tabref{tab:right} のように個々の図表と各々の \|\caption|/\|\ecaption| 
を \|minipage| 環境に入れることで実現できます.
なお図と表が混在する場合,\|minipage| 環境の中で\|\CaptionType{figure}| あるいは \|\CaptionType|\|{table}| を指定すれば,外側の環境が \|figure| であっても \|table| であっても指定された見出しが得られます.

2段の幅にまたがる図は,\figref{fig:double} の形式で指定します.
位置の指定は \|t| しか使えません.

図の中身では本文と違い,どのような大きさのフォントを使用しても構いません(\figref{fig:double} 参照).
また図の中身として,encapsulate されたPostScriptファイル(いわゆるEPSファイル)を読み込むこともできます.
読み込みのためには,プリアンブルで
%
\begin{quote}
\|\usepackage{graphicx}|
\end{quote}
%
を行った上で,\|\includegraphics| コマンドを図を埋め込む箇所に置き,その引数にファイル名(など)を指定します.

\subsection{表}

表の罫線はなるべく少なくするのが,仕上がりをすっきりさせるコツです.
罫線をつける場合には,一番上の罫線には二重線を使い,左右の端には縦の罫線をつけません (\tabref{tab:example}).
表中のフォントサイズのデフォルトは\|\footnotesize|です.

また,表の上に和文と英文の双方の見出しを, \|\caption|と \|\ecaption| で指定します.
表の参照は \|\tabref{<|ラベル\|>}| を用いて行ないます.

\begin{table}[tb] 
\caption{表の例} 
\ecaption{An Example of Table.}
\label{tab:example}
\hbox to\hsize{\hfil
\begin{tabular}{l|lll}\hline\hline
& column1 & column2 & column3 \\\hline
row1 &	item 1,1 & item 2,1 & ---\\
row2 &	---      & item 2,2 & item 3,2 \\
row3 &	item 1,3 & item 2,3 & item 3,3 \\
row4 &	item 1,4 & item 2,4 & item 3,4 \\\hline
\end{tabular}\hfil}
\end{table}

\subsection{参考文献・謝辞}

\subsubsection{参考文献の参照}

本文中で参考文献を参照する場合には\|\cite|を使用します.
参照されたラベルは自動的にソートされ,\|[]|でそれぞれ区切られます.
%
\begin{quote}
文献 \|\cite{companion,okumura}| は \LaTeX の総合的な解説書である.
\end{quote}
%
と書くと;
%
\begin{quote}
文献\cite{companion,okumura}は \LaTeX の総合的な解説書である.
\end{quote}
%
が得られます.

\subsubsection{参考文献リスト}
参考文献リストには,原則として本文中で引用した文献のみを列挙します.
順序は参照順あるいは第一著者の苗字のアルファベット順とします.
文献リストはBiB\TeX と\verb+ipsjunsrt.bst+(参照順)または\verb+ipsjsort.bst+(アルファベット順)を用いて作り,\verb+\bibliograhpystyle+と\verb+\bibliography+コマンドにより利用することが出来ます.
これらを用いれば,規定の体裁にあったものができるので,できるだけ利用してください.

\subsubsection{謝辞}

謝辞がある場合には,参考文献リストの直前に置き,\|acknowledgment|環境の中に入れます.

\section{おわりに}

\cssname スタイルファイル (\|css.cls|) の使い方を説明しました.
基本的な使い方は情報処理学会の標準スタイルファイル (\|ipsj.cls|) に準じます.

もし明白な \|css.cls| のバグ,もしくは本サンプルファイルの誤りを発見した場合は,\cssname 事務局 (\cssemail) までご連絡ください.

\begin{acknowledgment}
本サンプルファイルの作成にあたっては,情報処理学会の論文執筆キットのサンプルファイルから大幅に抜粋しました.
ここに感謝します.
\end{acknowledgment}

\begin{thebibliography}{10}

\bibitem{okumura}
奥村晴彦:改訂第5版 \LaTeXe 美文書作成入門,
技術評論社(2010).

\bibitem{companion}
Goossens, M., Mittelbach, F. and Samarin, A.:
{\it The LaTeX Companion},
Addison Wesley, Reading, Massachusetts (1993).

%\bibitem{webpage1}
%情報処理学会論文誌ジャーナル編集委員会:
%投稿者マニュアル(online),
%\urlj{http://www.ipsj.or.jp/journal /submit/manual/j\_manual.html}
%(2007.04.05).

\end{thebibliography}

\appendix

\section{付録の書き方}

付録がある場合には,参考文献リストの直後にコマンド \|\appendix| に引き続いて書きます.
付録では,\|\section| コマンドが{\bf A.1},{\bf A.2}などの見出しを生成します.

%7.1
\subsection{見出しの例}

付録の \|\subsetion| ではこのような見出しになります.

\end{document}
